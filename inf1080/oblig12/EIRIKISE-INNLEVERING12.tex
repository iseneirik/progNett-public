\documentclass[norsk, 12p]{article}
\usepackage[utf8]{inputenc}
\usepackage{babel}
\usepackage{amsmath}
\usepackage[top = 1cm]{geometry}
\usepackage{enumerate}

\setcounter{secnumdepth}{0}

\title{Obligatorisk Innlevering}
\author{Eirik Isene}

\begin{document}
\maketitle

\subsection{Oppgave 20.15}
\begin{enumerate}[a)]
    \item Ikke en ekvivalensrelasjon, mangler $\langle e,e\rangle$
    \item Er en ekvivalensrelasjon, $[a] = \{a,b\}$
    \item Ikke en ekvivalensrelasjon, mangler $\langle c,d\rangle$ og $\langle d,c\rangle$
    \item Ikke en ekvivalensrelasjon, mangler $\langle c,a\rangle$, $\langle c,b\rangle$, $\langle a,b\rangle$ og $\langle b,a\rangle$
    \item Er en ekvivalensrelasjon, $[a] = \{a\}$
    \item Er en ekvivalensrelasjon, $[a] = \{a,c\}$
\end{enumerate}

\subsection{Oppgave 20.16}
\subsubsection{Oppgave a)} 
\begin{enumerate}   
    \item La $\sim$ være en ekvivalensrelasjon på de naturlige tallene
    \item La $E$ være $[0]$
    \item Vis at $E \neq \emptyset$
    \item Siden $\sim$ er en ekvivalensrelasjon (1), er det en refleksiv relasjon
    \item Siden $\sim$ er en refleksiv relasjon (4), så er $\langle 0,0\rangle \in \sim$
    \item Siden $\langle 0,0\rangle \in \sim$ (5), så er $0 \in [0]$
    \item Dermed er det vist at $E \neq \emptyset$ fordi $E = [0]$ (2) og $[0] \neq \emptyset$ (6) 
\end{enumerate}
\subsubsection{Oppgave b)}
\begin{enumerate} 
    \item La $\sim$ være en ekvivalensrelasjon på de naturlige tallene
    \item La $E$ være $[0]$
    \item La $x, y \in E$
    \item Siden $x, y \in E$ (3) så er $\langle 0,x\rangle ,\langle 0,y\rangle \in \sim$
    \item Siden $\langle 0,x\rangle \in \sim$ (4) så følger det av symmetri at $\langle x,0\rangle \in \sim$ ($\sim$ er en ekvivalensrelasjon (1))
    \item siden $\langle x,0\rangle \in \sim$ (5) og $\langle 0,y\rangle \in \sim$ (4) så følger det av transitivitet at $\langle x,y\rangle \in \sim$ ($\sim$ er en ekvivalensrelasjon (1))
    \item dermed er det bevist at alle naturlige tall $x$ og $y$ som er elementer i $E$ er det slik at $x \sim y$ (6)
\end{enumerate}
\subsubsection{Oppgave c)}
Ja, definisjonen av en ekvivalensklasse sier at alle elementer i en ekvivalensklasse er relatert til hverandre i ekvivalensrelasjonen som skaper ekvivalensklassene, da kan man ut fra definisjonen si at dersom $[x] = [y]$ så vil $x \sim y$ også være tilfelle!

\subsection{Oppgave 21.16}
Det finnes $5^5 = 3125$ funksjoner fra $S$ til $S$, og $5! = 120$ av disse er bijeksjoner.


\end{document}
