\documentclass[norsk, 12p]{article}
\usepackage[utf8]{inputenc}
\usepackage{babel}
\usepackage{amsmath}
\usepackage[top = 1cm]{geometry}
\usepackage{enumerate}

\setcounter{secnumdepth}{0}

\title{Obligatorisk Innlevering 11}
\author{Eirik Isene}

\begin{document}
\maketitle

\subsection{Oppgave 19.12}
\begin{enumerate}[a)]
    \item $\{4,5\}$
    \item $\{\emptyset,\{1\},\{4\},\{1,4\}\}$
    \item $\{\{a,b,c,d\},\{e,f\}\}$
    \item $\{\{1,3\},\{2,4\}\}$
    \item fordi $\{1,2\}\cap\{2,3\} \neq \emptyset$
    \item Ja! alle mengder er delmengder i seg selv!
\end{enumerate}

\subsection{Oppgave 19.13}
\begin{itemize}
    \item[a.] $R$ er transitiv og symmetrisk på $S$
    \item[b.] For alle $x \in S$ finnes det en $y$ i $S$ slik at $Rxy$
    \item[ * ] En ekvivalensrelasjon er transitiv, symmetrisk og refleksiv, dermed må vi vise at $R$ er refleksiv!
\end{itemize}
\begin{enumerate}
    \item Anta $a \in S$, vis $Raa$
    \item Siden det finnes en $b$ slik at $Rab$ (antakelse b) så medfører det av symmetri (antakelse a) at $Rba$
    \item Siden $Rab$ (2) og $Rba$ (2) så medfører det av tranistivitet (antakelse a) at $Raa$
    \item Siden $Raa$ (3) for en vilkårlig $a \in S$ så har vi bevist at $R$ er refleksiv for alle $a \in S$, og dermed en ekvivalensrelasjon!
\end{enumerate}


\end{document}
