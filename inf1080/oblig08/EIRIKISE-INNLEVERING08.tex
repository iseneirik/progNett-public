\documentclass[norsk, 12p]{article}
\usepackage[utf8]{inputenc}
\usepackage{color}
\usepackage{babel}
\usepackage{amsmath}
\usepackage{amsfonts}
%\usepackage[top = 1cm]{geometry}

\newcommand{\hl}[1]{\textcolor{blue}{#1}}

\setcounter{secnumdepth}{0}

\title{Obligatorisk Innlevering 05}
\author{Eirik Isene}

\begin{document}
\maketitle

\subsection{Oppgave 13.4}
Vis ved induksjon at følgende påstand er sann for alle naturlige tall n.
$$n^3 - n \text{ er delelig med 3}$$

\subsubsection{Basissteget}
Vi sier at $P(n)$ står for at $n^3 - n = 3a$ er sant, hvor $a \in \mathbb{N}$ \\ 
Vi må vise at påstanden stemmer for $n = 0$:
$$0^3 - 0  =  0$$
$3a = 0 \Rightarrow a = 0$ og $0 \in \mathbb{N}$, P(0) er sann! \\
Vi antar så at det stemmer for P(n) altså at:
$$n^3 - n = 3a$$
Hvor $a \in \mathbb{N}$, dette er \hl{induksjonshypotesen} vår

\subsubsection{Induksjonssteget}
Vi må så vise at påstanden stemmer for $(n+1)$: \\ \\
\begin{tabular}{r l l}
$P(n+1)$ & $=(n + 1)^3 - (n + 1)$             & | Skal være delelig med $3$ \\ 
         & $=(n^3 + 3n^2 + 3n + 1) - (n + 1)$ & | Regning \\
         & $=n^3 + 3n^2 + 3n + 1 - n - 1$     & | Regning \\
         & $=n^3 + 3n^2 + 3n - n$             & | Regning \\
         & $=\hl{n^3 - n} + 3n^2 + 3n$        & | \hl{Induksjonshypotesen} \\
         & $=\hl{3a} + 3n^2 + 3n$             & | $a \in \mathbb{N}$  \\
         & $=3(a + n^2 + n)$                  & | $3$ er en faktor i alle ledd \\
\end{tabular} \\ \\
Siden $a \in \mathbb{N}$ og $n \in \mathbb{N}$ så har vi vist at påstanden stemmer for $P(n+1)$!

\pagebreak

\subsection{Oppgave 14.6}
\subsubsection{Oppgave C}
\begin{enumerate}
\item $f(f(b)) = b$
\item Induksjon
\item Induksjonshypotesen
\item $f(f(bx))$
\item $bx$
\item Punkt 2
\item $f(b)0$
\item Punkt 3
\item Strukturell induksjon på mengden av binære tall
\item $f(f(b)) = b$
\end{enumerate}

\end{document}
