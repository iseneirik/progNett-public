\documentclass[norsk, 12p]{article}
\usepackage[utf8]{inputenc}
\usepackage{babel}
\usepackage{amsmath}
\usepackage[top = 1cm]{geometry}

\setcounter{secnumdepth}{0}

\title{Obligatorisk Innlevering 04}
\author{Eirik Isene}

\begin{document}
\maketitle

\section{Oppgave 7.7}
Hvis $(P \to Q)$ er sann og $(Q \to R)$ er sann, så er $(P \to R)$ sann

\subsection{Oppgave a)}
Ved direkte bevis:
\begin{description}
\item[(1)] Vi antar at $(P \to Q)$ er sann og at $(Q \to R)$ er sann, og skal vise at $(P \to R)$ er sann
\item[(2)] Vi antar at $P$ er sann, for dersom $P$ er usann, blir $(P \to R)$ alltid sann.
\item[(3)] Dersom $(P \to Q)$ er sann $(1)$, og $P$ er sann $(2)$, må $Q$ være sann.
\item[(4)] Dersom $(Q \to R)$ er sann $(1)$ og $P$ er sann $(2)$ og $Q$ er sann $(3)$, må $R$ være sann
\item[(5)] Dersom $P$ er sann $(2)$ og $R$ er sann $(4)$, så er $(P \to R)$ sann
\end{description}

\subsection{Oppgave c)}
Ved et motsigelsesbevis:
\begin{description}
\item[(1)] Vi antar at $(P \to R)$ er usann
\item[(2)] Vi antar at $(P \to Q)$ er sann og $(Q \to R)$ er sann
\item[(3)] Dersom $(P \to R)$ er usann $(1)$, må $P$ være sann og $R$ være usann
\item[(4)] Dersom $P$ er sann $(3)$ og $(P \to Q)$ er sann $(2)$, må $Q$ være sann
\item[(5)] Dersom $Q$ er sann $(4)$ og $R$ er usann $(3)$ så er $(Q \to R)$ usann
\item[(6)] Vi ser at $(Q \to R)$ er usann $(5)$ og $(Q \to R)$ er sann $(2)$, dette er en motsigelse.   
\end{description}
Siden vi ender i en motsigelse ved å anta det motsatte, er den originale antakelsen sann!

\section{Oppgave 7.10}
Er påstanden sann eller usann? $F$ står for en utsagnslogisk formel.

\subsection{Oppgave g)}
For alle $F$, så er $F$ oppfyllbar eller $\neg F$ oppfyllbar.
\begin{description}
\item[(1)] Dersom $F$ er en kontradiksjon vil $\neg F$ være en tautologi, altså oppfyllbar.
\item[(2)] Dersom $F$ ikke er en kontradiksjon, så er den oppfyllbar.
\end{description}

\subsection{Oppgave h)}
For alle $F$, så er $F$ en tautologi eller $\neg F$ en tautologi.
\begin{description}
\item[(1)] Dersom $F$ er både oppfyllbar og falsifiserbar er verken $F$ en tautologi, eller $\neg F$ en tautologi.
\end{description}

\end{document}
