\documentclass[norsk, 12pt]{article}
\usepackage[utf8]{inputenc}
\usepackage{babel}
\usepackage{amsmath}
\usepackage[top=1cm]{geometry}

\setcounter{secnumdepth}{0}

\title{Obligatorisk Innlevering 03}
\author{Eirik Isene}

\begin{document}
\maketitle

\section{Oppgave 4.13}
\begin{align*}
&(A \land B)& &\Leftrightarrow& &\neg(\neg A \lor \neg B)& &\Leftrightarrow& &\neg(A \to \neg B) \\
&(A \land \neg B)& &\Leftrightarrow& &\neg(\neg A \lor B)& &\Leftrightarrow& &\neg(A \to B) \\
&(\neg A \land B)& &\Leftrightarrow& &\neg(A \lor \neg B)& &\Leftrightarrow& &\neg(\neg A \to \neg B) \\
&(\neg A \land \neg B)& &\Leftrightarrow& &\neg(A \lor B)& &\Leftrightarrow& &\neg(\neg A \to B) \\
\end{align*}

\section{Oppgave 5.3}
\begin{description}
\item[(a)] $(P \lor Q) \to \neg P$ \\ Oppfyllbar når: P er usann \\ Falsifiserbar når: P er sann
\item[(b)] $P \lor (Q \to \neg P)$ \\ Tautologi fordi $(Q \to \neg P)$ blir sann alle ganger P er usann og de står på hver sin side av $\lor$ tegnet
\item[(c)] $(P \land Q) \to \neg P$ \\ Oppfyllbar når: P er usann \\ Falsifiserbar når: P og Q er sann
\item[(d)] $((P \to Q) \land \neg Q) \to \neg P$ \\ Tautologi, fordi om man prøver å falsifisere, må P være usann (så $\neg P$ blir sann) og videre må Q være sann (så $(P \to Q)$ blir sann), men da blir $\neg Q$ usann, som gjør hele $((P \to Q) \land \neg Q)$ usann, og dermed blir utrykket sant
\item[(e)] $\neg(P \lor Q)\land (\neg Q \lor R) \land (\neg R \lor P)$ \\ Oppfyllbar når: P, Q og R er usann \\ Falsifiserbar ved alle andre kombinasjoner av P og Q
\item[(f)] $(\neg(P \lor Q)) \land P$ \\ Kontradiksjon fordi P må være usann for at $\neg(P \lor Q)$ skal være sann, og dermed kan formelen aldri valueres til sann fordi P kan ikke være sann og usann samtidig!
\end{description}

\section{Oppgave 5.8}
For å lage en eksklusiv formel, altså at formelen blir sann når, og bare når en av variablene er sann og ikke den andre, så kommer vi fra til denne formelen: \\ \\
$(\neg F \land G) \lor (F \land \neg G)$  \\ \\
Dette blir riktig fordi første del av formelen blir sann kun når $G$ er sann og $F$ er usann, og andre del av formelen blir sann kun når $F$ er sann og $G$ er usann, siden disse to formlene er kombinert sammen i en eller formel, så blir hele formelen sann når enten bare $F$ eller bare $G$ er sann.

\end{document}
